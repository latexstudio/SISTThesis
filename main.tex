% !TeX program = xelatex
% !TeX encoding = UTF-8
%  请使用xeLaTeX进行编译
\documentclass[postdoc,twoside]{sistthesis} % replace postdoc by master or doctor
%\usepackage[numbers,comma, sort&compress]{natbib}      % 引用格式使用数字  [1,2] 
\usepackage[super,square,comma,sort&compress]{natbib} % 引用格式使用上标  ^{[1,2]}
\bibliographystyle{gbt-7714-2015-numerical} %使用GBT7714标准的参考文献格式 (按引用顺序排列参考文献)
%\bibliographystyle{gbt-7714-2015-author-year} %使用GBT7714标准的参考文献格式(按作者-日期排列参考文献)

\usepackage{hypernat} 
\usepackage{subfig}
\usepackage{comment,multirow,float}
% set search path
\graphicspath{{pic/}}




%%%%% 分类号等
\classification{TM13}
\confidential{不涉密}
\UDC{}
\IDnumber{}

%%%%%%%%%%%%%%%%%%%%%%%%%%%%%%%%%%%%%%%%%%%%%%%%%%%%%%%%%%%%%%%%%%%%%%%%%%%%%%%%%%%%%
%%%%%%%%%%%  博士后工作报告将此处注释取消
%%%%%   封面
\title{博士后工作报告\LaTeX 模板}
\author{王官杰}
\jobbegin{2015年7月22日}
\jobend{2018年3月31日}
\reportfinish{2015年7月—2018年3月}
\submitdate{2018年3月} 

%%%%% 题名页
\englishtitle{\LaTeX\ Template of Postdoctoral Research Report}
\discipline{电子科学与技术}
\major{电磁场与微波技术}
\institute{中国科学院上海微系统与信息技术研究所}
\address{上海}
%%%%%%%%%%%%%%%%%%%%%%%%%%%%%%%%%%%%%%%%%%%%%%%%%%%%%%%%%%%%%%%%%%%%%%%%%%%%%%%%%%%%%%%%%%%%%%%%



%%%%%%%%%%%%%%%%%%%%%%%%%%%%%%%%%%%%%%%%%%%%%%%%%%%%%%%%%%%%%%%%%%%%%%%%%%%%%%%%%%%%%
%%%%%%%%%%%  学位论文将此处注释取消
%%%%    中文题名页
%\title{带随机变量的Helmholtz方程的高效数值方法}
%\setitle{Efficient Numerical methods for}
%\setitlee{Helmholtz equations with random inputs}
%\author{王官杰}
%\discipline{电子科学与技术}
%\major{电磁场与微波技术}
%\research{微分方程数值解}
%\advisor{朱建新}
%\school{信息科学与技术学院}
%\submitdate{2018年3月} 

%%%%%     英文题名页
%\englishtitle{Efficient numerical methods for\\ Helmholtz equations with random inputs}
%\englishauthor{Guanjie Wang}
%\englishschool{School of Inforamtion Science and Technology}
%\englishsubmitdate{Apr., 2017}

%%%%%%%%%%%%%%%%%%%%%%%%%%%%%%%%%%%%%%%%%%%%%%%%%%%%%%%%%%%%%%%%%%%%%%%%%%%%%%%%%%%%%%%%%%%%%%

% 自定义宏命令,不需要可删除,也可自行添加需要的宏命令
\newtheorem{definition}{定义}[chapter]
\newtheorem{example}{例}[chapter]
\newtheorem{lemma}{引理}[chapter]
\newtheorem{theorem}{定理}[chapter]
\newtheorem{remark}{注}[chapter]


\begin{document}

	
\maketitle
\makeenglishtitle

\frontmatter
%\include{chap/announce}
\begin{abstract}
	各位老师,同学,上午好!
	很高兴Adobe软件开通下载受到了大家的欢迎。但由于昨天下载流量过高导致服务器难以负荷,为保证下载效率,目前已限制最多10人同时登录下载。若登录下载人数已超限,将出现无法登录的情况,建议错开下载高峰。
	非常感谢老师同学们的谅解与支持!
	
	\keywords{Helmholtz方程,完美匹配层,传播计算,算子步进方法,局部基变换,共轭算子,弯曲界面,局部正交坐标变换}
\end{abstract}

\begin{englishabstract}





\englishkeywords{Helmholtz equation, PML, wave propagation, operator marching method, local base transformation, adjoint operator, curved interface, local orthogonal coordinate transformation}

\end{englishabstract}
\tableofcontents
\listoffigures
\listoftables
\include{chap/notation}


%%正文部分
\mainmatter
\include{chap/intro}
\chapter{正交多项式}
\include{chap/math}
\chapter{表格图形}\label{sec:tabfig}

\section{表格}
\section{图形}
\appendix
\include{chap/req}


\backmatter
\include{chap/thanks}
\bibliography{book,journal}
\addcontentsline{toc}{chapter}{\bibname}
\include{chap/resume}
\begin{publications}{99}
\item J. Zhu and {\bf G. Wang}. 
``Fast computation of wave propagation in the open acoustical waveguide with a curved interface",  {\em Wave motion}, {\bf 57}, 171-181, 2015.
\item J. Zhu and {\bf G. Wang}. 
``New computational treatment of optical wave propagation in lossy waveguides'',  {\em Frontiers of Information Technology \& Electronic Engineering}, {\bf 16}(8), 646-653, 2015.
\item  J. Zhu and {\bf G. Wang}.
``High-precision computation of optical propagation
in inhomogeneous waveguides'',  {\em Journal of the Optical Society of America A}, {\bf 32}(9), 1653-1660, 2015.


\end{publications}


\end{document}